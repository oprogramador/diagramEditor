
%%%%%%%%%%%%%%%%%%%%%%%%%%%%%%%%%%%%%%%%%
%
% Copyright Piotr Sroczkowski
%
%%%%%%%%%%%%%%%%%%%%%%%%%%%%%%%%%%%%%%%%%

%----------------------------------------------------------------------------------------
%	PACKAGES AND DOCUMENT CONFIGURATIONS
%----------------------------------------------------------------------------------------

\documentclass{article}

\usepackage[version=3]{mhchem} % Package for chemical equation typesetting
\usepackage{siunitx} % Provides the \SI{}{} and \si{} command for typesetting SI units
\usepackage{graphicx} % Required for the inclusion of images
\usepackage{natbib} % Required to change bibliography style to APA
\usepackage{amsmath} % Required for some math elements 
\usepackage{datetime}
\usepackage[utf8]{inputenc}
\usepackage[polish]{babel}
\usepackage{polski}


\setlength\parindent{0pt} % Removes all indentation from paragraphs

\renewcommand{\labelenumi}{\alph{enumi}.} % Make numbering in the enumerate environment by letter rather than number (e.g. section 6)

%\usepackage{times} % Uncomment to use the Times New Roman font

%----------------------------------------------------------------------------------------
%	DOCUMENT INFORMATION
%----------------------------------------------------------------------------------------

\title{Przedmiot specjalnościowy\\ Programowanie interfejsu użytkownika internetu\\ Edytor } % Title


\author{Piotr \textsc{Sroczkowski}} % Author name

\date{\today} % Date for the report

\begin{document}

\maketitle % Insert the title, author and date


\begin{center}
\includegraphics{logo.png}
\begin{tabular}{l r}
Prowadzący: & dr inż Marek Żabka % Instructor/supervisor
\end{tabular}
\end{center}

\pagebreak

% If you wish to include an abstract, uncomment the lines below
% \begin{abstract}
% Abstract text
% \end{abstract}

%----------------------------------------------------------------------------------------
%	SECTION 1
%----------------------------------------------------------------------------------------


\section{Założenia projektu strony}
\begin{enumerate}
    \item Dodawanie elementu do diagramu
    \item Przesuwanie elementu
    \item Sprawdzanie kolizji z innymi elementami i granicami diagramu przy przesuwaniu elementu
    \item Sprawdzanie kolizji przy dodawaniu nowego elementu
    \item Zmiana tekstu, koloru, kształtu elementu
    \item Łączenie elementów, usuwanie połączeń
    \item Usuwanie elementu, czyszczenie diagramu
    \item Eksport do formatu graficznego
\end{enumerate}

\end{document}
